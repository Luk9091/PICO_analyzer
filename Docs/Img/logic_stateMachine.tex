\begin{figure}[!ht]
    \centering
    \begin{circuitikz}
        \draw
            (0,  0) node[draw, circle, align=center, minimum width = 3cm](trig){Oczekiwanie\\ na wyzwolenie}
            (4, -4) node[draw, circle, align=center, minimum width = 3cm](sample){Zebranie\\próbki}
            (0, -8) node[draw, circle, align=center, minimum width = 3cm](lock){Blokada\\wyzwalania}
            (8, -8) node[draw, circle, align=center, minimum width = 3cm](dma){Wyzwolenie\\DMA}
            (6, -12) node[draw, circle, align=center, minimum width = 3cm, dashed](time){Zapisanie\\czasu}
            (0, -12) node[draw, circle, align=center, minimum width = 3cm](store){Zapisanie\\próbki}
        ;

        \draw[very thick, -Stealth] (trig) to[bend left] node[right]{$trigger = 1$} (sample);
        \draw[very thick, -Stealth] (trig) to[loop left] (trig);
        \draw[very thick, -Stealth] (sample) to[bend left] (lock);
        \draw[very thick, -Stealth] (lock) to[bend left] node[left]{$trigger = 0$} (trig);

        \draw[very thick, -Stealth] (sample) to[bend right] (dma);
        \draw[very thick, -Stealth] (dma) to[bend left] (time);
        \draw[very thick, -Stealth] (time) to[short] (store);
    \end{circuitikz}
    \caption{Maszyna stanów, odpowiedzialna za akwizycję danych}
    \label{fig:logic_stateMachine}
\end{figure}