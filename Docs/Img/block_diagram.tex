\begin{figure}[!ht]
    \centering
    \begin{circuitikz}
        \draw
            (0, 0) node[draw, minimum width = 6cm, minimum height = 3cm, align=center](PICO){\LARGE{Raspberry PI}\\\ \\\large{PICO W}}
            (3, -5)node[draw, minimum width = 12cm, minimum height = 3cm, align=center](PINS){\LARGE{Pin header}}

            (0, 4) node[draw, minimum width = 2cm, minimum height = 2cm](LDO){\LARGE{LDO}}
            % (5, 2) node[draw, minimum width = 2cm, minimum height = 2cm](V_REF){\LARGE{$\text{V}_{\text{ref}}$}}
            (6.5, 4) node[draw, minimum width = 2cm, minimum height = 2cm](ADC){\LARGE{ADC}}
            (5, 0) node[draw, minimum width = 2cm, minimum height = 2cm](AAF_PICO){\LARGE{AAF}}
            (8, 0) node[draw, minimum width = 2cm, minimum height = 2cm](AAF_ADC){\LARGE{AAF}}


            (PICO.west) to[short, -o] ++(-2, 0) node[above]{\large{USB}}
            (PICO.south) to[tmultiwire, l2=digital and logic, a=16] ++(0, -2)
            (LDO.south) to[short, a=$5V$] (PICO.north)

            (PICO.east) to[bmultiwire, l = 2] (AAF_PICO.west)
            (LDO) to[short, l=$3.3V$] (ADC) -| (8, 4) to[bmultiwire, l=2] (AAF_ADC.north)
            (ADC.south) --++ (0, -0.75) to[bmultiwire, a=$I^2C$] ++ (-5, 0) -- ++(0, -0.75)

            (AAF_PICO.south) to[bmultiwire, l=2] ++ (0, -2.5) 
            (AAF_ADC.south)  to[bmultiwire, l=2] ++ (0, -2.5)
        ;
    \end{circuitikz}
    \caption{Schemat blokowy analizatora stanów logicznych}
    \label{figure:block_diagram}
\end{figure}