\section{Implementacja algorytmów komunikacji analizatora z komputerem użytkownika.}
% Tutaj Łukasz możesz też opisać implementacji protokołu po USB
Analizator do komunikacji wykorzystuje dwie metody transmisji: WIFI oraz USB.   

\subsection{Komunikacja bezprzewodowa - WIFI}

\indent Do komunikacji bezprzewodowej wykorzystano dostępny na raspberry pi Pico
układ CYW43439. Jako protokół transmisyjny wykorzystan UDP ze względu na mały narzut
obliczeniowy oraz mniejszą latencję, dodatkowo aby współpraca z urządzeniem była maksymalnie prosta ustawiono Pi Pico
jako punkt dostępu(access point) oraz skonfigurowano na nim serwer DHCP który przydziela adresy
IP podłączającym sie do niego użytkownikom. Schemat działania protokołów transmisji
odstępnych na pi pico przedstawiono poniżej.


\begin{figure}[ht]
\centering
\begin{tikzpicture}[
  device/.style={rectangle, draw, rounded corners, minimum width=4cm, minimum height=1.2cm, align=center, font=\small},
  arrow/.style={-{Latex}, thick},
  bidir/.style={<->, thick, dashed}, % przerywana linia dwukierunkowa
  every node/.style={font=\small}
]

% Komputer użytkownika
\node[device, fill=blue!10] (pc) {Komputer użytkownika\\(dedykowana aplikacja)};

% RP2040 jako AP
\node[device, fill=green!20, right=6cm of pc] (pico) {RP2040 (Pico W)\\jako Access Point};

% Dwukierunkowa przerywana strzałka z podpisem (Wi-Fi)
\draw[bidir] (pc) -- ++(0,1) -- ++(10,0) -- ++(0,-0.4) node[below right, xshift=-6.5cm, yshift=1cm] {UDP przez Wi-Fi} (pico);

% Połączenie USB (ciągła linia) - poprawione, strzałka do pico
\draw[->, <->, thick] (pc) -- ++(0,-1) -- ++(10,0) -- ++(0,0.4) node[below right, xshift=-6.5cm, yshift=-0.5cm] {Połączenie USB} (pico);

\end{tikzpicture}
\caption{Rozwiązanie kwestii transmisji w analizatorze}
\label{fig:udp-komunikacja}
\end{figure}



%\subsection{Komunikajca przewodowa - USB}
%...
%TODO