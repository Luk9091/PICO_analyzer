\section{Opis i założenia projektu}
    Celem projektu jest zaprojektowanie oraz zbudowanie systemu analizatora stanów logicznych.
    System analizatora składa się z 2 części:
    \begin{itemize}
        \item Układu analizatora - dedykowane PCB zawierające min. mikrokontroler
        Raspberry Pi Pico W, przetwornik ADC, stabilne źródło napięcia odniesienia(bandgap),
        układ filtracji sygnału analogowego(LPF) oraz interfejs wejściowy wraz z zabezpieczeniami
        chroniącymi układ przed przypadkowymi pomyłkami użytkownika.
        \item Aplikacji komputerowej - interfejs użytkownika odpowiadający za intuicyjną komunikację 
        między użytkownikiem a urządzeniem. Aplikacja komputerowa stworzona zostanie z wykorzystaniem
        framework-u GTK4 oraz języka C. GUI dodatkowo będzie wykorzystywać mechanizmy wielowątkowości w celu
        zapewnienia odpowiedniej kultury pracy z aplikacją. 
    \end{itemize}

\subsection{Założenia sprzętowe}
    \begin{enumerate}
        \item Analizator stanów logicznych, jako układ pomiarowy nie powinien obciążać urządzenia badanego dlatego
        system pomiarowy analizatora będzie pracował w trybie wysokiej impedancji wejściowej.
        \item Układ powinien charakteryzować się co najmniej dobrymi parametrami takimi jak:
        szybkość pracy, rozdzielczość sygnałów, dokładność oraz precyzja pomiaru.
        \item System powinien umożliwiać pomiary sygnałów których częstotliwość bitowa będzie wynosić co najmniej
        100kB/s %tak se podałem te 100 bo nie wiem ile ma mieć 
    \end{enumerate}

\subsection{Założenia oprogramowania}
    \begin{enumerate}
        \item Aplikacja komputerowa powinna stanowić wygodny oraz intuicyjny interfejs między użytkownikiem
        a urządzeniem.
        \item Oprogramowanie mikrokontroler w maksymalnym stopniu powinno opierać się na zasobach
        hardware-owych takich jak DMA, Timery, akceleratory IO(PIO) itp. .
    \end{enumerate}
    
\subsection{Podział obowiązków}
    \textbf{Łukasz Przystupa}
    \begin{enumerate}
        \item Stworzenie aplikacji komputerowej w GTK4
        \item Zaprojektowanie niskopoziomowego systemu próbkującego opartego o PIO.
        \item Konfiguracja bardzo szybkiej i stabilnej komunikacji szeregowej między
        komputerem użytkownika a PicoProbe z wykorzystaniem interfejsu USB.
    \end{enumerate}

    \textbf{Krzysztof Płonka}
    \begin{enumerate}
        \item Napisanie biblioteki do obsługi czujnika ADC ADS1115.
        \item Zapewnienie komunikacji bezprzewodowej między Pi Pico a aplikacją graficzną
        \item Dodanie funkcjonalności obserwacji sygnałów analogowych
        (bibliotek gtk live-chart, integracja z SigRok lub biblioteka CAIRO).
    \end{enumerate}

    \textbf{Paweł Olbrych}
    \begin{enumerate}
        \item Projekt PCB.
        \item Dobór elementów.
        \item Zaprojektowanie min.: interfejsu wejściowego układu analizatora stanów,
        interfejsu wejściowego przetworników ADC, układu filtracji oraz polaryzacji zasilania oraz
        mniejszych podobwodów pomocniczych. 
    \end{enumerate}
    