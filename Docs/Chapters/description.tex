\section{Opis i założenia projektu}
    Celem projektu jest zaprojektowanie oraz zbudowanie analizatora stanów logicznych z dedykowaną aplikacją graficzną.
    Urządzenie urządzenie podłączane będzie do komputera za pomocą USB.
    Z drugiej strony dla użytkownika zostane oddane listwa kołkowa umożliwiający podłączyć sygnałów analizowanych.

\subsection{Założenia sprzętowe}
    Analizator stanów logicznych, jako układ pomiarowy nie powinien obciążać urządzenia badanego.
    W związku z czym układ cyfrowy powinien mieć możliwość ustawienia wyprowadzeń w stan wysokiej impedancji.

    Kolejnym ograniczeniem jest maksymalna częstotliwość pracy układu powinna być nie mniejsza niż częstotliwość próbkowania.
    W przypadku bardzo wysokich częstotliwości wejściowych, dane powinny być buforowane.
    % TODO: Nie wiem wymyślcie

\subsection{Założenia oprogramowania}
    Projekt zakłada zbudowanie dwóch programów.
    Pierwszą będzie firmware mikrokontrolera, będącego serem części hardwarowej.
    Drugą zaś aplikacja graficzna pozwalająca wyświetlić użytkownikowi dane w wygodny sposób.

    \subsubsection{Oprogramowanie mikroprocesora}
        Mikrokontroler w maksymalnym stopniu powinien opierać się na zasobach hardwarowych, takich jak DMA czy akceleratory IO.
        Natomiast komunikacja z komputerem, powinna być możliwie prosta - przykładowo wykorzystując protokół UART.
        A w znacznie lepszym wykorzystaniem byłoby standard CDC (Communication Device Class) -- wirtualne połączenie przez port szeregowy,
        dzięki czemu można przesyłać dane z maksymalną przepustowością \textit{USB 2.0}.


\subsection{Podział obowiązków}
    \begin{itemize}
        \item Łukasz Przystupa  - napisanie aplikacji graficznej oraz oprogramowanie Raspberry PI PICO.
        \item Krzysztof Płonka  - napisanie komunikacji przez Wi-Fi oraz zgodności z aplikacją ???. %nie wiem jak ona się nazywa
        \item Paweł Olbrych     - zaprojektowanie PCB.
    \end{itemize}