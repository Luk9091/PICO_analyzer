\section{Opis i założenia projektu}
    Celem projektu jest zaprojektowanie oraz zbudowanie systemu analizatora stanów logicznych.
    System analizatora składa się z 2 części:
    \begin{itemize}
        \item Układu analizatora - dedykowane PCB zawierające min. mikrokontroler
        Raspberry Pi Pico W, przetwornik ADC, stabilne źródło napięcia odniesienia(bandgap),
        układ filtracji sygnału analogowego(LPF) oraz interfejs wejściowy wraz z zabezpieczeniami
        chroniącymi układ przed przypadkowymi pomyłkami użytkownika.
        \item Aplikacji komputerowej - interfejs użytkownika odpowiadający za intuicyjną komunikację 
        między użytkownikiem a urządzeniem. Aplikacja komputerowa stworzona zostanie z wykorzystaniem
        framework-u GTK4 oraz języka C. GUI dodatkowo będzie wykorzystywać mechanizmy wielowątkowości w celu
        zapewnienia odpowiedniej kultury pracy z aplikacją. 
    \end{itemize}

\subsection{Założenia sprzętowe}
    \begin{enumerate}
        \item Analizator stanów logicznych, jako układ pomiarowy nie powinien obciążać urządzenia badanego dlatego
        system pomiarowy analizatora będzie mieć możliwość ustawienia wyprowadzeń w stan wysokiej impedancji.
        \item Ukłąd powinien charakteryzować się co najmniej dobrymi parametrami takimi jak:
        szybkość pracy, rozdzielczość sygnałów, dokładność oraz precyzja pomiaru.
        \item System powinien umożliwiać pomiary sygnałów których częstotliwość bitowa będzie wynosić co najmniej
        100kB/s %tak se podałem te 100 bo nie wiem ile ma mieć 
    \end{enumerate}

    section{Założenia oprogramowania}
    \begin{enumerate}
        \item Aplikacja komputerowa powinna stanowić wygodny oraz intuicyjny interfejs między użytkownikiem
        a urządzeniem.
        \item Oprogramowanie mikrokontroler w maksymalnym stopniu powinno opierać się na zasobach
        hardware-owych takich jak DMA, Timery, akceleratory IO(PIO) itp. .
    \end{enumerate}
    
% TO bym dał potem a w wstępie dał tylko założenia
% \subsection{Założenia oprogramowania}
%     \begin{enumerate}
%         \item Projekt zakłada stworzenie dwóch aplikacji.
%         Pierwszą z nich będzie oprogramowanie mikrokontrolera, będącego sercem części hardwarowej.
%         Drugą zaś aplikacja graficzna pozwalająca wyświetlić użytkownikowi dane w wygodny sposób.
%     \end{enumerate}

% \subsubsection{Oprogramowanie mikroprocesora}
%     \begin{enumerate}
%         \item Mikrokontroler w maksymalnym stopniu powinien opierać się na zasobach hardwareowych, takich jak DMA czy akceleratory IO.
%         Natomiast komunikacja z komputerem, powinna być możliwie prosta - przykładowo wykorzystując protokół UART.
%         A w znacznie lepszym wykorzystaniem byłoby standard CDC (Communication Device Class) -- wirtualne połączenie przez port szeregowy,
%         dzięki czemu można przesyłać dane z maksymalną przepustowością \textit{USB 2.0}.
%     \end{enumerate}


\subsection{Podział obowiązków}
    \begin{itemize}
        \item Łukasz Przystupa  - napisanie aplikacji graficznej oraz oprogramowanie Raspberry PI PICO.
        \item Krzysztof Płonka  - stworzenie interfejscu do komunikacji bezprzewodowej(Wi-Fi),
        biblioteki do układu ADC oraz (opcjonalne) ... . % TODO
        \item Paweł Olbrych     - zaprojektowanie PCB.
    \end{itemize}