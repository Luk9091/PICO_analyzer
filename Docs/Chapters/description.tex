\section{Opis i założenia projektu}
    Celem projektu jest zaprojektowanie oraz zbudowanie systemu analizatora stanów logicznych.
    System analizatora składa się z 2 części:
    \begin{itemize}
        \item Układu analizatora - dedykowane PCB zawierające min. mikrokontroler
        Raspberry Pi Pico W, przetwornik ADC, układ filtracji sygnału analogowego(LPF)
        oraz interfejs wejściowy wraz z zabezpieczeniami chroniącymi układ przed przypadkowymi
        pomyłkami użytkownika.
        \item Aplikacji komputerowej - interfejs użytkownika odpowiadający za intuicyjną
        komunikację między użytkownikiem a urządzeniem. 
        % Aplikacja komputerowa stworzona zostanie z wykorzystaniem framework-u GTKmm oraz języka C++.
        % GUI będzie wykorzystywać mechanizmy wielowątkowości w celu zapewnienia odpowiedniej kultury pracy z aplikacją. 
        W celu zapewnienia responsywności, aplikacja powinna dzielić wyświetlanie oraz przetwarzanie danych na oddzielne wątki.
    \end{itemize}

\subsection{Założenia sprzętowe}
    \begin{enumerate}
        \item Analizator stanów logicznych, jako układ pomiarowy nie powinien obciążać prądowo
        urządzenia badanego dlatego system będzie pracował w trybie
        wysokiej impedancji wejściowej.
        \item Podstawowymi parametrami branymi urządzenia są:
        \begin{itemize}
            \item wysoką szybkością próbkowania sygnałów cyfrowych,
            \item odpowiednią rozdzielczością pomiarowego toru analogowego,
            \item stabilność pracy,
        \end{itemize}
        \item Maksymalna częstotliwość próbkowania sygnałów cyfrowych powinna wynosić co najmniej $f_{\text{sample}} = 50MHz$.
    \end{enumerate}

\subsection{Założenia oprogramowania}
    \begin{enumerate}
        \item Aplikacja komputerowa powinna stanowić wygodny oraz intuicyjny interfejs
        między użytkownikiem a urządzeniem.
        \item Oprogramowanie mikrokontrolera w maksymalnym stopniu powinno opierać się na
        zasobach sprzętowych układu takich jak DMA, timery, akceleratory IO (PIO) itp. .
    \end{enumerate}
    
\subsection{Podział obowiązków}
    \textbf{Łukasz Przystupa}
    \begin{enumerate}
        \item Stworzenie aplikacji komputerowej.
        \item Zaprojektowanie niskopoziomowego systemu próbkującego sekcji cyfrowej opartego o PIO wbudowany w Raspberry PI PICO.
        \item Konfiguracja bardzo szybkiej i stabilnej komunikacji szeregowej między
        komputerem użytkownika a PicoProbe z wykorzystaniem interfejsu USB.
    \end{enumerate}

    \textbf{Krzysztof Płonka}
    \begin{enumerate}
        \item Napisanie biblioteki do obsługi czujnika ADC ADS1115.
        \item Zapewnienie komunikacji bezprzewodowej między Pi Pico a aplikacją graficzną
        \item Dodanie funkcjonalności obserwacji sygnałów analogowych.
        (bibliotek gtk live-chart, integracja z SigRok lub biblioteka CAIRO)\footnote{Opcjonalnie.}.\\ 
    \end{enumerate}

    \textbf{Paweł Olbrych}
    \begin{enumerate}
        \item Projekt PCB.
        \item Dobór elementów.
        \item Zaprojektowanie min.: interfejsu wejściowego układu analizatora stanów,
        interfejsu wejściowego przetworników ADC, układu filtracji oraz polaryzacji zasilania oraz
        mniejszych podobwodów pomocniczych. 
    \end{enumerate}
    