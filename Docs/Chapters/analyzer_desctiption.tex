\section{Analiza sygnałów logicznych}
    Analizator stanów cyfrowych w całości oparty jest na peryferiach wbudowanych w mikrokontroler.
    Jego najważniejszą część stanowi akcelerator IO, będący niewielką hardwarową maszyną stanów, 
    która umożliwia niezależnie od rdzeni wykonywać instrukcje sterujące portem wejścia-wyjścia,
    co w połączeniu z układami DMA, pozwala na pewną autonomię działania analizatora stanów logicznych.


\subsection{Implementacja analizatora stanów}
    Na rysunku \ref{fig:logic_stateMachine}, przedstawiono budowę automatu próbkującego dla sygnałów cyfrowych.

    \begin{figure}[!ht]
    \centering
    \begin{circuitikz}
        \draw
            (0,  0) node[draw, circle, align=center, minimum width = 3cm](trig){Oczekiwanie\\ na wyzwolenie}
            (4, -4) node[draw, circle, align=center, minimum width = 3cm](sample){Zebranie\\próbki}
            (0, -8) node[draw, circle, align=center, minimum width = 3cm](lock){Blokada\\wyzwalania}
            (8, -8) node[draw, circle, align=center, minimum width = 3cm](dma){Wyzwolenie\\DMA}
            (6, -12) node[draw, circle, align=center, minimum width = 3cm, dashed](time){Zapisanie\\czasu}
            (0, -12) node[draw, circle, align=center, minimum width = 3cm](store){Zapisanie\\próbki}
        ;

        \draw[very thick, -Stealth] (trig) to[bend left] node[right]{$trigger = 1$} (sample);
        \draw[very thick, -Stealth] (trig) to[loop left] (trig);
        \draw[very thick, -Stealth] (sample) to[bend left] (lock);
        \draw[very thick, -Stealth] (lock) to[bend left] node[left]{$trigger = 0$} (trig);

        \draw[very thick, -Stealth] (sample) to[bend right] (dma);
        \draw[very thick, -Stealth] (dma) to[bend left] (time);
        \draw[very thick, -Stealth] (time) to[short] (store);
    \end{circuitikz}
    \caption{Maszyna stanów, odpowiedzialna za akwizycję danych}
    \label{fig:logic_stateMachine}
\end{figure}

    Układ może być wyzwalany zewnętrzem sygnałem, przychodzącym na jedno z wejść cyfrowych lub na specjalnie wyznaczone wejście wyzwalające opisane jako $trig0$ lub $trig 1$.
    Alternatywnym sposobem wyzwalania próbkowania jest wykorzystanie wewnętrznego taktowania Raspberry PI PICO, dzięki czemu, można regularnie analizować stany z częstotliwością od $5kHz$ do maksymalnie $200MHz$.

    Dodatkową opcją, która może okazać się przydatna, jest pomiar czasu pomiędzy kolejnymi impulsami wyzwalającymi.
    Wykorzystuje ona jeden z kanałów PWM, jako licznik zliczający impulsy zegara - co pozwala mierzyć czas z maksymalną precyzją na poziomie $5ns$.
    Mechanizm ten wykorzystuje priorytetyzację modułów DMA, przez co maksymalna częstotliwość próbkowania spada do $50MHz$.


\subsection{Buforowanie w pamięci - synchronizacja DMA}
    Moduł analizatora wykorzystuje łącznie 4 z 12 dostępnych kanałów DMA, po dwa na zapisywanie próbek logicznych i dwa na zapisywanie czasu.
    Każda z par połączona jest w topologii typu ,,\textit{Ping-Pong}" dzięki czemu, kiedy jeden kanał zapełni się, natychmiast uruchamiany jest jego odpowiednik.
    Takie połączenie, umożliwia ciągłą (pierścieniową) praca układu próbkującego.

    Wszystkie cztery kanały wyzwalane są tym samym żądaniem ,,\textit{data request (DREQ)}".
    Wadą takiego rozwiązania jest synchronizacja procesu.
    Ponieważ zgłoszenie ,,\textit{DREQ}" jest rozgłoszeniowe, co w domyślnej konfiguracji, uniemożliwia określenie, która para zostanie wywołana jako pierwsza, przez co  może to powodować błędy zapisu czasu.
    Rozwiązaniem jest jawne nadanie wyższego priorytetu, dla pary obsługującej licznik.
    Poniżej przedstawiono graf pracy układów DMA w wyżej opisanym trybie.
    \begin{figure}[!ht]
    \centering
    \begin{circuitikz}
        \draw
            ( 0, 0) node[draw, circle, align=center, minimum width=3cm](dreq){Wyzwolenie\\DMA}
            ( 4, 0) node[draw, circle, align=center, minimum width=3cm](time){Zapis\\czasu}
            ( 8, 0) node[draw, circle, align=center, minimum width=3cm](sample){Zapis\\danych}
            (12, 0) node[draw, circle, align=center, minimum width=3cm](end){Zdjęcie\\flagi\\,,\textit{DREQ}''}
        ;

        \draw[very thick, -Stealth] (dreq) -- (time);
        \draw[very thick, -Stealth] (time) -- (sample);
        \draw[very thick, -Stealth] (sample) -- (end);
    \end{circuitikz}
    \caption{Łańcuch wyzwoleń kanałów DMA}
    \label{fig:dma_routine}
\end{figure}
    



    % Co więcej kanały odpowiedzialne za zapisywanie czasu, 

