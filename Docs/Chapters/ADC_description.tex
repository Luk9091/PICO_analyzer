\section{Implementacja algorytmów przetwarzania sygnałów z przetworników analogowo-cyfrowych}
Analizator wykorzystuje dwa przetworniki ADC:
\begin{itemize}
    \item Wewnętrzny - przetwornik SAR(Successive Approximation) o parametrach
        \begin{itemize}
            \item Typ : SAR
            \item Wykonanie : jeden przetwornik z multipleksowanym wejściem(4 wejścia) 
            \item $f_{smax}$ = 500 kHz (przypadek dla jednego kanału)
            \item Rozdzielczość : 12 bit (8.7 ENOB)
            \item Tryb pracy: pojedynczy / różnicowy
            \item $V_{inmax}$ = 3.3 V
        \end{itemize}
        
    \item Zewnętrzny - przetwornik $\Delta\Sigma$ (Delta Sigma)
        \begin{itemize}
            \item Typ : Sigma - Delta
            \item Wykonanie : jeden przetwornik z multipleksowanym wejściem(4 wejścia) 
            \item $f_{smax}$ = 860 Hz (przypadek dla jednego kanału)
            \item Rozdzielczość : 16 bit 
            \item Tryb pracy: pojedynczy (brak możliwości badania sygnałów różnicowych) 
            \item $V_{inmax}$ = 6.144 V
        \end{itemize}
\end{itemize}

\subsection{Wewnętrzny przetwornik ADC(SAR)}
\subsubsection{Konfiguracja.}
Przetwornik skonfigurowano w następujący sposób:
    \begin{itemize}
        \item $f_{s}$ = 5 kHz (z racji wykorzystania dwóch kanałów: $f_{sch1}$ = 2.5 kHz, $f_{sch1}$ = 2.5 kHz)
        \item Wyzwalanie konwersji : wyzwalanie za pomocą timera. 
    \end{itemize}

\subsubsection{Zasadaa działania.}
Wewnętrzny przetwornik działa w oparciu o dwa bufory, które na zmianę zapełnia danymi. Takie 
działanie pozwala przetwarzać(np. wysłać) ostatnio zebrane dane które przez okres 
zapełniania drugiego bufora nie będą nadpisywane.\\
\indent Wyzwalaniem przetwornika zajmuje się timer który odpytuje go o nowe dane z zadaną częstotliwością.
Działanie przetwornika można by w znacznym stopniu zoptymalizować przez skonfigurowanie go w 
trybie kołowym oraz wykorzystaniu DMA. W takim przypadku przetwornik sam zbierałby dane ze swojego
wejścia, przełączał na kolejne a to co zebrał zapisywał do dedykowanego rejestru z którego dane
były by odbierane przez DMA. Zaprojektowana biblioteka do obsługi ADC uwzględnia taką możliwość
jednak w aktualne wersji analizatora jest ona wyłączona na rzecz odpytywania przetwornika z poziomu
przerwania od timera. Głównym powodem takiego wyboru jest fakt dostępności tylko jednego 
DMA(multipleksowanego na 12 kanałów), oraz priorytetowej roli części cyfrowej(analizatora stanów
logicznych).


\subsection{Zewnętrzny przetwornik ADC(ADS1115)}
\begin{itemize}
    \item $f_{s}$ = 860 kHz(z racji wykorzystania dwóch kanałów: $f_{sch1}$ = 430, $f_{sch1}$ = 430)
    \item Wyzwalanie konwersji : wyzwalanie za pomocą timera. 
    \item Komunikacja : I2C (prędkość: 400kHz)
\end{itemize}

\indent Zewnętrzny przetwornik podobnie jak ten dostępny w RP2040 również został skonfigurowany do cyklicznego 
odpytywania go o nowe dane przez Timer, oraz podobnie jak poprzednio jego praca opiera się o 
naprzemienne zapełnianie dwóch buforów(tym razem buforów kołowych). \\
\indent Do komunikacji z mikrokontrolerem ADS1115 wykorzystuje interfejs I2C skonfigurowane
z prędkością 400 kHz.

\begin{figure}[ht]
    \centering
    \begin{tikzpicture}[
      buffer/.style={rectangle, draw, minimum width=2.5cm, minimum height=1cm, text centered},
      arrow/.style={-{Latex}, thick},
      every node/.style={font=\small}
    ]
    
    % ADC
    \node[buffer, fill=blue!10] (adc) {ADC};
    
    % Buffers
    \node[buffer, fill=yellow!30, right=3cm of adc] (bufA) {Bufor A};
    \node[buffer, fill=orange!30, below=2cm of bufA] (bufB) {Bufor B};
    
    % Arrows from ADC
    \draw[arrow] (adc) -- node[above] {Próbkowanie} (bufA);
    
    % Processing
    \node[buffer, fill=green!20, right=2cm of bufA] (procA) {Przetwarzanie / Wysyłanie};
    
    \draw[arrow] (bufB) -- node[below, sloped] {wysyłanie} (procA);
    %\draw[arrow] (bufA) -- (procA);
    
    % Swapping arrows (loop)
    \draw[arrow, dashed] (bufA.south) to[bend right=20] node[left] {\shortstack{Zmiana\\buforów}} (bufB.north);
    \draw[arrow, dashed] (bufB.north) to[bend right=20] node[right] {} (bufA.south);
  
    \end{tikzpicture}
    \caption{Mechanizm podwójnego buforowania próbek z ADC}
    \label{fig:double-buffering}
  \end{figure}