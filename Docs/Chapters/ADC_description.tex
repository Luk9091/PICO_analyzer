\section{Przetwornik analogowo-cyfrowy}
Analizator wykorzystuje dwa przetworniki ADC:
\begin{itemize}
    \item Wewnętrzny - przetwornik SAR (Successive Approximation) o parametrach:
        \begin{itemize}
            % \item Typ : SAR
            \item wykonanie: jeden przetwornik z multipleksowanym wejściem (4 wejścia + wewnętrzny pomiar temperatury),
            \item $f_{s_{max}}$ = 500 kHz (dla jednego kanału),
            \item rozdzielczość: 12 bit (8.7 ENOB),
            \item tryb pracy: pojedynczy (brak możliwości badania sygnałów różnicowych),
            \item $V_{in_{max}}$ = 3.3 V,
        \end{itemize}
        
    \item Zewnętrzny - przetwornik $\Delta\Sigma$ (Delta Sigma) - układ ADC1115
        \begin{itemize}
            % \item Typ : Sigma - Delta
            \item wykonanie: jeden przetwornik z multipleksowanym wejściem (4 wejścia),
            \item $f_{s_{max}}$ = 860 Hz (dla jednego kanału),
            \item rozdzielczość : 16 bit ,
            \item tryb pracy: pojedynczy/różnicowy,
            \item $V_{in_{max}}$ = $2.048V$ - z możliwości regulacji programowej (PGA),
        \end{itemize}
\end{itemize}

\subsection{Wewnętrzny przetwornik ADC (SAR)}
\subsubsection{Konfiguracja.}
Przetwornik wewnętrzny skonfigurowano w następujący sposób:
    \begin{itemize}
        \item $f_{sampling_{ch1}}$ = $5kHz$, 
        \item $f_{sampling_{ch2}}$ = $5kHz$,
        \item wyzwalanie konwersji: wyzwalanie za pomocą timera.
    \end{itemize}

\subsubsection{Zasada działania.}
    Wewnętrzny przetwornik działa w oparciu o dwa bufory, które na zmianę zapełnia danymi. Takie 
    działanie pozwala przetwarzać (np. wysłać) ostatnio zebrane dane które przez okres 
    zapełniania drugiego bufora nie będą nadpisywane.

    Wyzwalaniem przetwornika zajmuje się timer, który odpytuje go o nowe dane z zadaną częstotliwością.
    Działanie przetwornika można by w znacznym stopniu zoptymalizować przez skonfigurowanie go w trybie kołowym oraz wykorzystaniu DMA. 
    W takim przypadku przetwornik sam zbierałby dane ze swojego wejścia następnie przełączał na kolejne a to co zebrał zapisywał do dedykowanego rejestru z którego dane były by odbierane przez DMA.
    Zaprojektowana biblioteka do obsługi ADC uwzględnia taką możliwość, jednak w aktualnej wersji analizatora jest ona wyłączona na rzecz odpytywania przetwornika z poziomu przerwania od timera. 
    Głównym powodem takiego wyboru jest fakt dostępności tylko jednego DMA (multipleksowanego na 12 kanałów) oraz priorytetowej roli części cyfrowej (analizatora stanów logicznych).


\subsection{Zewnętrzny przetwornik ADC(ADS1115)}
Przetwornik zewnętrzny skonfigurowano w następujący sposób:
\begin{itemize}
    \item $f_{sampling_{ch1}}$ = $500Hz$, 
    \item $f_{sampling_{ch2}}$ = $500Hz$,
    \item wyzwalanie konwersji: wyzwalanie za pomocą timera. 
    \item komunikacja: $I^2C$ (prędkość $400kHz$)
\end{itemize}

Zewnętrzny przetwornik podobnie jak ten dostępny w RP2040 również został skonfigurowany do cyklicznego odpytywania
o nowe dane przez licznik (timer) oraz podobnie jak poprzednio jego praca opiera się o naprzemienne zapełnianie dwóch buforów kołowych. \\
% Do komunikacji z mikrokontrolerem, ADS1115 wykorzystuje interfejs I2C skonfigurowany z prędkością 400 kHz.

\subsection{Schemat blokowy analizatora stanów analogowych.}

\begin{figure}[ht]
    \centering
    \begin{tikzpicture}[
      buffer/.style={rectangle, draw, minimum width=2.5cm, minimum height=1cm, text centered},
      arrow/.style={-Stealth, thick},
      every node/.style={font=\small}
    ]
    
    % ADC
   \node[buffer, fill=blue!30, align=center] (adc) {ADS1115 ADC 3\\ ADS1115 ADC 4};
    
    % Buffers
    \node[buffer, fill=blue!30, right=3cm of adc] (bufA) {Bufor A};
    \node[buffer, fill=blue!30, below=2cm of bufA] (bufB) {Bufor B};
    
    % Arrows from ADC
    \draw[arrow] (adc) -- node[above] {Próbkowanie} (bufA);
    
    % Processing
    \node[buffer, fill=blue!30, right=2cm of bufA] (procA) {Przetwarzanie / Wysyłanie};
    
    \draw[arrow] (bufB.east) -| ++(4.43cm, 1cm) -| (procA.south); % wiem że wartość 4.43 to
   %lipa ale nie miałem czasu rozkminia o co chodzi, przepraszam
    %\draw[arrow] (bufA) -- (procA);
    
    % Swapping arrows (loop)
    \draw[arrow, dashed] (bufA.south) to[bend right=20] node[left] {\shortstack{Zmiana\\buforów}} (bufB.north);
    \draw[arrow, dashed] (bufB.north) to[bend right=20] node[right] {} (bufA.south);
  
    \end{tikzpicture}
    \caption{Mechanizm podwójnego buforowania próbek z ADC ADS1115}
    \label{fig:double-buffering-ADS1115}
  \end{figure}




  \begin{figure}[ht]
    \centering
    \begin{tikzpicture}[
      buffer/.style={rectangle, draw, minimum width=2.5cm, minimum height=1cm, text centered},
      arrow/.style={-Stealth, thick},
      every node/.style={font=\small}
    ]
    
    % ADC
   \node[buffer, fill=blue!30, align=center] (adc) {Pico ADC 0\\Pico ADC 1};
    
    % Buffers
    \node[buffer, fill=blue!30, right=3cm of adc] (bufA) {Bufor A};
    \node[buffer, fill=blue!30, below=2cm of bufA] (bufB) {Bufor B};
    
    % Arrows from ADC
    \draw[arrow] (adc) -- node[above] {Próbkowanie} (bufA);
    
    % Processing
    \node[buffer, fill=blue!30, right=2cm of bufA] (procA) {Przetwarzanie / Wysyłanie};
    
    \draw[arrow] (bufB.east) -| ++(4.43cm, 1cm) -| (procA.south); % wiem że wartość 4.43 to
   %lipa ale nie miałem czasu rozkminia o co chodzi, przepraszam
    %\draw[arrow] (bufA) -- (procA);
    
    % Swapping arrows (loop)
    \draw[arrow, dashed] (bufA.south) to[bend right=20] node[left] {\shortstack{Zmiana\\buforów}} (bufB.north);
    \draw[arrow, dashed] (bufB.north) to[bend right=20] node[right] {} (bufA.south);
  
    \end{tikzpicture}
    \caption{Mechanizm podwójnego buforowania próbek z ADC Pico}
    \label{fig:double-buffering-Pico}
  \end{figure}