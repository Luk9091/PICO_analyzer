\section{Analizator sygnałów analogowych - test zaprojektowanego systemu}
    Kolejnym etapem projektowania układu do pomiaru sygnałów analogowych było przetestowanie 
    zaprojektowanego systemu.

\subsection{Opis stanowiska pomiarowego}
    Pomiary wykonano z użyciem generatora \textit{DDS FG-100 DDS FUNCTION GENERATOR}
    oraz przenośnego oscyloskopu \textit{Fnirsi 1c15} jako układu odniesienia z pomocą którego
    zadawano parametry generowanych sygnałów. Stanowisko pomiarowe przedstawiono poniżej.

    \begin{figure}[!ht]
        \centering
        \includegraphics[width = \textwidth]{stanowisko_pomiarowe.png}
        \caption{Układ stanowiska pomiarowego}
        \label{fig:stanowisko_pomiarowe}
    \end{figure} 

    W celu przeprowadzenia testów działania systemu, które na wczesnym etapie projektu
    mogłyby się odbywać niezależnie od aplikacji graficznej, postanowiono napisać
    skrypt w języku Python, dzieki któremu możliwe było obserwowanie zebranych danych
    wysyłanych przez WiFi do komputera. Widok aplikacji przedstawiono poniżej.
    
    \begin{figure}[H]
        \centering
        \includegraphics[width = \textwidth]{Analog_track_2.png}
        \caption{Pomocnicza aplikacja do obserwowania przebiegów sygnałów analogowych}
        \label{fig:Analog_track_2}
    \end{figure}

\subsection{Testowanie działania analizatora sygnałów analogowych}
    Jednym z najważniejszych parametrów przy testowaniu układów takich jak oscyloskopy
    czy ogólnie analizatory sygnałów analogowych jest m. in. maksymalna częstotliwość
    sygnałów, które wiernie mogą odwzorowywać. Częstotliwość ta wynika bezpośrednio z 
    twierdzenia o próbkowaniu Nyquista-Shannona:
    \begin{equation*}
        f_s \geq 2 f_{\max}
    \end{equation*}

    W rzeczywistości jednak częstotliwość próbkowania powinna być większa niż 
    dwukrotność maksymalnej składowej częstotliwościowej sygnału, często podaje się ją
    jako:
    \begin{equation*}
        f_s \geq 2.5 f_{\max}
    \end{equation*}
    lub nawet więcej.

    Podczas przeprowadzenia testów, kryterium badania jakości mierzonych sygnałów była wartość
    THD (dla 4 harmonicznych), którą definiuje się następująco:
    \begin{equation*}
    \mathrm{THD}_{dB} = 20 \log_{10} \left( \frac{
    \sqrt{
    \sum_{h=2}^{5} \left|X(h f_1)\right|^2
    }
    }{
    \left|X(f_1)\right|
    } \right)
    \end{equation*}
    Jeżeli więc zniekształcenia osiągały duże wartości(duże THD, czyli dużą zawartość
    pozostałych harmonicznych względem podstawowej) stwierdzano, że dalsze zwiększanie częstotliwości nie ma sensu
    i kończono pomiar. Zebrane wyniki pomiarów przedstawiono poniżej.

    \begin{table}[!ht]
        \centering
        \begin{tabular}{|c|>{\centering\arraybackslash}m{4cm}|>{\centering\arraybackslash}m{4cm}|}
            \hline
            \textbf{Nazwa przetwornika} &
            \makecell{\textbf{Teoretyczna maks.}\\\textbf{częstotliwość}} &
            \makecell{\textbf{Zmierzona maks.}\\\textbf{częstotliwość}} \\
            \hline
            ADS1115 & $250 Hz$ & $180 Hz$ \\
            \hline
            Pico ADC & $2.5 kHz$ & $60 Hz$ \\
            \hline
        \end{tabular}
        \caption{Wyniki testów analizatora sygnałów analogowych}
        \label{tab:freq_test_results}
    \end{table}

    \subsection{Zaobserwowane problemy z działaniem układu}
    Jak można było to już wcześniej zauważyć (rys.~\ref{fig:Analog_track_2})
    wykres amplitudowy z ADC Pico jest mocno zniekształcony, jeszcze lepiej
    widać to na poniższej ilustracji.
     
    \begin{figure}[H]
        \centering
        \includegraphics[width = \textwidth]{analog_distortion.png}
        \caption{Wstępnie przetworzony sygnał wyjściowy z ADC Pico}
        \label{fig:analog_distortion}
    \end{figure}

    Charakter wykresu nie jest spowodowany zbyt wysoką częstotliwością podawanego
    sygnału lecz skutkiem działania systemu. Duże zniekształcenia sygnału wynikają
    z transmisji $I^2C$, która z racji na bardzo sprzętowy charakter, jeżeli wystąpi blokuje
    cały rdzeń w skutek tego zarejestrowany przebieg jest znacznie zniekształcony.
    Podczas projektowania układu podejmowano różne techniki niwelowania takiego zachowania tj.: 
    \begin{enumerate}
        \item Praca z $I^2C$, na przerwaniach z wykorzystaniem maszyny stanów.
        \item Wykorzystanie wbudowanej funkcjonalności ADS1115, która umożliwia zgłaszanie przez
        przetwornik zdarzenia zakończenia konwersji. 
        Dzięki zastosowaniu przerwania od tego sygnału, procesor nie musi czekać na zakończenie próbkowania.
        %  przez co nie trzeba na nią czekać tracąc czas rdzenia.
    \end{enumerate}

    Innym pomysłem na rozwiązanie tego problemu było by zastosowanie innego przetwornika ADC wykorzystującego szybsze protokoły transmisji (np. SPI)
    lub inne topologie wewnętrznego ADC.

    Niestety ostatecznie nie udało się doprowadzić żadnej z wyżej wymienionych opcji do stanu,
    który można by uznać za zadowalający, głównie ze względu na rosnącą złożoność sytemu kontrolno-pomiarowego,
    którego dobrze działająca implementacja okazała się bardzo pracochłonna i skomplikowana.
    
    Dlatego właśnie maksymalna częstotliwość Pico ADC pomimo znacznie większego
    próbkowania ostatecznie okazała się sporo mniejsza niż znacznie wolniejszego ADS1115.  

    Warto również zauważyć, że częstotliwość została podana orientacyjnie ponieważ ze względu
    na okresowy charakter pracy $I^2C$ można znaleźć częstotliwość gdzie sinus jest odwzorowywany bardzo dobrze.
    Z przeprowadzonych testów częstotliwość ta wynosi:
    \begin{align}
        f_{aliasing} = 310 \cdot n,\ n \in \mathbb{N}
    \end{align}