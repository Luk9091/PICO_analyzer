\section{Komunikacji analizatora z komputerem użytkownika}
Analizator do komunikacji wykorzystuje dwie metody transmisji: WIFI lub USB.

\begin{figure}[ht]
    \centering
    \begin{tikzpicture}[
        device/.style={rectangle, draw, rounded corners, minimum width=4cm, minimum height=1.2cm, align=center, font=\small},
        arrow/.style={-Stealth, very thick},
        bidir/.style={Stealth-Stealth, thick, dashed}, % przerywana linia dwukierunkowa
        every node/.style={font=\small}
    ]

        % Komputer użytkownika
        \node[device, fill=blue!10] (pc) {Komputer użytkownika\\(dedykowana aplikacja)};

        % RP2040 jako AP
        \node[device, fill=green!20, right=6cm of pc] (pico) {RP2040 (Pico W)\\jako Access Point};

        \draw[bidir] (pc) -- ++(0,1) -- ++(10,0) -- ++(0,-0.4) node[below right, xshift=-6.5cm, yshift=1cm] {UDP przez Wi-Fi} (pico);
        \draw[-Stealth, Stealth-Stealth, very thick] (pc) -- ++(0,-1) -- ++(10,0) -- ++(0,0.4) node[below right, xshift=-6.5cm, yshift=-0.5cm] {Połączenie USB} (pico);

    \end{tikzpicture}
    \caption{Rozwiązanie kwestii transmisji w analizatorze}
    \label{fig:udp-komunikacja}
\end{figure}

\subsection{Komunikacja bezprzewodowa - WIFI}
    Do komunikacji bezprzewodowej wykorzystano układ CYW43439, dostępny na Raspberry PI Pico. 
    Jako protokół transmisyjny wykorzystan UDP ze względu na mały narzut
    obliczeniowy oraz mniejszą latencję, dodatkowo aby współpraca z urządzeniem była maksymalnie prosta ustawiono Pi Pico
    jako punkt dostępu(access point) oraz skonfigurowano na nim serwer DHCP który przydziela adresy
    IP podłączającym sie do niego użytkownikom. Schemat działania protokołów transmisji
    odstępnych na pi pico przedstawiono poniżej.


\subsection{Komunikacja przewodowa - USB}
    Układ RP2040 jak i RP2350 wyposażone są w kontrolery $USB 1.2$ umożliwiające transmisję z maksymalną prędkością $12Mbps$.
    Jednak aby wykorzystać maksymalną przepustowość kanału, niezbędne jest stworzenie dodatkowego sterownika, po stronie komputera odbiorczego.
    Aby maksymalnie uprościć proces przesyłania danych, oraz zapewnić jak największą szybkość transmisji, kontroler USB został ustawiony w tryb CDC (Communication Device Class),
    pozwalający na emulację komunikacji szeregowej.

    Tak zaprogramowany kontroler, pozwalana na prostą komunikację za pomocą portu szeregowe, z szybkością nieporównywalna z klasycznymi układami \textit{USB UART}.
    W tabli poniżej przedstawiono uśredniony czas wysłania wiadomości ,,Hello, world$\backslash$n$\backslash$r" w pętli 500 razy, za pomocą różnych rozwiązań:
    \begin{table}[!ht]
        \centering
        \caption{Średni czas wysłania wiadomości za pomocą różnych instrukcji}
        \begin{tabular}{|c|c|c|c|}\hline
            Protokół      & Polecenie wysłania & Zmierzony czas & Szybkość transmisji\\
            komunikacyjny & danych & $[ms]$& $[kbps]$\\\hline
            USB & tud\_cdc\_write &   1 520 & 294.74  \\\hline
            USB & printf          &   7 624 &  58.76  \\\hline
            UART& uart\_putc      & 604 732 &   0.74 \\\hline
            UART& printf          & 656 235 &   0.68 \\\hline
        \end{tabular}
    \end{table}

    Jak widać wykorzystanie sprzętowego kontrolera USB niesie ze sobą olbrzymią przewagę.
    A dodatkowe wyeliminowanie, funkcji formatującej pozwala przyspieszyć prędkość ponad pięciokrotnie!


\subsection{Ramka transmisyjna}
    Powyższa analiza pozwoliła, zbudować warstwę fizyczną (\citetitle{Model_TCP_IP} \cite{Model_TCP_IP}).
    Następnym krokiem, powinna być (zgodnie z modelem TCP/IP) warstwa internetowa, jednak w przypadku sieci PAN, ten problemem rozwiązuje się samoistnie.
    Także ostatnim krokiem poprzedzającym powstanie interfejsu użytkownika, zostaje zdefiniowanie ramki transmisyjnej.

    \subsubsection{Wielkość ramki danych}
        Wybór szerokości ramki, okazuje się nie trywialnym zadaniem i jest próbą znalezienia złotego środka między ograniczeniami sprzętowymi (pamięć RAM mikrokontrolera),
        czy wymogami standardów komunikacji (np. czas dostępu do okna transmisyjnego).

        W przypadku transmisji USB zgodnie z dokumentacją standardu \citetitle{USB20} \cite{USB20} maksymalna długość ramki jest ograniczona 64 bajtów.
        Co oznacza, że wszystkie ramki dłuższe niż 64 bajty, będą przez kontroler USB rozdzielane na paczki po 64 bajty.
        Z tego względu, dobrym pomysłem wydaje się ramka o rozmiarze będącym całkowitą wielokrotnością 64.

        Inna sytuacja występuje podczas przesyłania danych bezprzewodowo.
        Ograniczeniem transmisji nie jest głębokość rami (od 0 do nawet 2304 bajtów danych \citetitle{WiFi} \cite{WiFi}),
        natomiast synchroniczność przesyłania danych w ściśle określonych oknach czasowych.

        Podsumowując sensownym wydaję się ramka zawierająca między 256 a 512 bajtów.
    
    \subsection{Pola danych}
        Po określeniu rozmiaru danych, należy ustalić sposób ułożenia danych.
        Proponowanym przez autorów rozwiązanie przedstawia rysunek \ref{fig:frame}.

        \begin{figure}[!ht]
            \centering
            \begin{circuitikz}
                \draw
                    (0, 0) node [draw, rectangle, minimum width = 2cm, minimum height = 2cm, align = center]{Tag\\2 bajty}
                    (2, 0) node [draw, rectangle, minimum width = 2cm, minimum height = 2cm, align = center]{Addition\\2 bajty}
                    (7, 0) node [draw, rectangle, minimum width = 8cm, minimum height = 2cm, align = center]{Payload\\252 - 508 bajtów}
                ;
            \end{circuitikz}
            \caption{Przykładowa ramka transmisyjna}
            \label{fig:frame}
        \end{figure}

        Opis pol w ramce
        \begin{itemize}
            \item Tag -- pole indetyfikacji, zawierające informację o tym co przenosi dana ramka,
            \item Addition -- pole wypełniające, pole zawierające dwa zerowe bajty, możliwe do wykorzystania w przyszłości,
            \item Payload -- pole z odpowiednimi danymi.
        \end{itemize}
    
    \subsection{Ułożenie danych}
        Moduł analizatora stanów logicznych pozwala na dużą dowolność w ustawionej konfiguracji.
        Użytkownik może między innymi ustawić zapis próbek cyfrowych wraz z sygnaturą.
        Wymusza to ograniczenia co do rozmiaru rozmiaru payload'u - jego głębokość musi być wielokrotnością 4 bajtów.
        Dzięki zastosowaniu tego ograniczenia, dane mogą być wysłane z przeplotem.
