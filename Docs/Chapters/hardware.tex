\section{Opis układów elektronicznych wykorzystanych w projekcie}

Do najważniejszych elementów elektronicznych wykorzystanych 
podczas projektowania analizatora należą min.:
 
\begin{enumerate}
    \item Raspberry Pi Pico W - platforma firmy Raspberry PI zawierająca 
    układ RP2040 oraz układ CYW43439. Rolę mikrokontrolera pełni
    tutaj układ z rodziny ARM Cortex M, a konkretnie ARM CORTEX M0+.
    Do komunikacji bezprzewodowej mikrokontroler wykorzystuje zewnętrzny 
    chip firmy Inineon o nazwie CYW43439. Układ ten pozwala na komunikację
    z wykorzystanie zarówno BLE jak i WIFI. \\
    Mikrokontroler dzięki obecności dedykowanego układu PIO pozwala na 
    pisanie bardzo szybkich prostych skryptów opartych o specjalne instrukcje assemblera,
    dzięki czemu można konstruować bardzo zoptymalizowane aplikacje wykorzystujące jego
    możliwości.   
    \item ADC ADS1115 - układ przetwornika analogowo-cyfrowego wyposażony  w przetwornik
    typu Sigma-Delta o rozdzielczości 16 bitów. Układ może działać z maksymalną
    prędkością 860 SPS(próbek na sekundę) a dzięki wykorzystaniu wbudowanego multipleksera
    może przełączać się między 4 wejściami single ended lub 2 różnicowymi. Przetwornik 
    został wyposażony w prosty interfejs komunikacyjny I2C za pomocą którego możliwa
    jest jego konfiguracja. 
    \item Stabilizator liniowy LDO ... TODO
    \item Zestaw 4 filtrów analogowych ... TODO  
    \item Diody zabezpieczające - diody zenera na napięcie 3.6V zabezpieczające
    stopień wejściowy analizatora przed przypadkowymi przepięciami i zwarciami.
    \item ... TODO
\end{enumerate}
