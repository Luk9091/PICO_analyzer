\section{Opis układów elektronicznych wykorzystanych w projekcie}

Do najważniejszych elementów elektronicznych wykorzystanych 
podczas projektowania analizatora należą min.:
 
\begin{enumerate}
    \item Raspberry Pi Pico W - platforma firmy Raspberry PI.
    % układ RP2040 oraz układ CYW43439. 
        Płytka wyposażona jest w mikrokontroler z rodziny ARM Cortex M0+, zaprojektowany przez firmę Raspberry Pi Foundation.
        Do komunikacji bezprzewodowej mikrokontroler wykorzystuje zewnętrzny chip firmy Infineon o nazwie CYW43439.
        Układ ten pozwala na komunikację z wykorzystanie zarówno BLE jak i WIFI.\\
        Jednym z peryferiów mikrokontrolera RP2040, są dwa układy PIO.
        Moduły te pozwalają na bardzo szybkie oraz niezależne od głównego programu zarządzanie układami IO,
        co w połączeniu z układami DMA, umożliwia próbkowanie i zapisywanie sygnałów wejściowych z częstotliwością pracy głównego rdzenia ($\approx 200MHz$).
    \item ADC ADS1115 - 
        układ przetwornika analogowo-cyfrowego typu Sigma-Delta o rozdzielczości 16 bitów.
        Układ może działać z maksymalną prędkością 860 sampli,
        a dzięki wykorzystaniu wbudowanego multipleksera może przełączać się między 4 wejściami single ended lub 2 różnicowymi.
        Przetwornik został wyposażony w prosty interfejs komunikacyjny I2C z pomocą, którego możliwa jest jego konfiguracja oraz przesył danych.
    \item Stabilizator liniowy LDO(TPS79333DBVR) na napięcie 3.3 V zapewniający stabilne
    zasilanie przetwornika analogowo-cyfrowego.
    \item Zestaw filtrów antyaliasingowych.
    \item Diody zabezpieczające(MM5Z3V6T1G) - diody Zenera na napięcie 3.6V zabezpieczające stopień wejściowy analizatora
    przed przypadkowymi przepięciami.
\end{enumerate}
